% https://medium.com/softway-blog/building-a-facial-recognition-machine-learning-model-using-tensorflow-6e62fb349794

%%%%%%%%%%%%%%%%%%%%%%%%%%%%%%%%%%%%%%%%
% datoteka diploma-vzorec.tex
%
% vzorčna datoteka za pisanje diplomskega dela v formatu LaTeX
% na UL Fakulteti za računalništvo in informatiko
%
% vkup spravil Gašper Fijavž, december 2010
% 
%
%
% verzija 12. februar 2014 (besedilo teme, seznam kratic, popravki Gašper Fijavž)
% verzija 10. marec 2014 (redakcijski popravki Zoran Bosnić)
% verzija 11. marec 2014 (redakcijski popravki Gašper Fijavž)
% verzija 15. april 2014 (pdf/a 1b compliance, not really - just claiming, Damjan Cvetan, Gašper Fijavž)
% verzija 23. april 2014 (privzeto cc licenca)
% verzija 16. september 2014 (odmiki strain od roba)
% verzija 28. oktober 2014 (odstranil vpisno številko)
% verija 5. februar 2015 (Literatura v kazalu, online literatura)
% verzija 25. september 2015 (angl. naslov v izjavi o avtorstvu)
% verzija 26. februar 2016 (UL izjava o avtorstvu)
% verzija 16. april 2016 (odstranjena izjava o avtorstvu)
% verzija 5. junij 2016 (Franc Solina dodal vrstice, ki jih je označil s svojim imenom)


\documentclass[a4paper, 12pt]{book}
%\documentclass[a4paper, 12pt, draft]{book}  Nalogo preverite tudi z opcijo draft, ki vam bo pokazala, katere vrstice so predolge!

\usepackage[utf8x]{inputenc}   % omogoča uporabo slovenskih črk kodiranih v formatu UTF-8
\usepackage[slovene,english]{babel}    % naloži, med drugim, slovenske delilne vzorce
\usepackage[pdftex]{graphicx}  % omogoča vlaganje slik različnih formatov
\usepackage{fancyhdr}          % poskrbi, na primer, za glave strani
\usepackage{amssymb}           % dodatni simboli
\usepackage{amsmath}           % eqref, npr.
%\usepackage{hyperxmp}
\usepackage[hyphens]{url}  % dodal Solina
\usepackage{comment}       % dodal Solina

\usepackage[pdftex, colorlinks=true,
						citecolor=black, filecolor=black, 
						linkcolor=black, urlcolor=black,
						pagebackref=false, 
						pdfproducer={LaTeX}, pdfcreator={LaTeX}, hidelinks]{hyperref}

\usepackage{color}       % dodal Solina
\usepackage{soul}       % dodal Solina
\usepackage[numbers]{natbib}  % dodal Solina
\usepackage{listings}
\usepackage[toc,page]{appendix}
\renewcommand{\appendixpagename}{Priloge}
\renewcommand{\appendixtocname}{Priloge}
\renewcommand{\appendixname}{Priloge}
%%%%%%%%%%%%%%%%%%%%%%%%%%%%%%%%%%%%%%%%
%	DIPLOMA INFO
%%%%%%%%%%%%%%%%%%%%%%%%%%%%%%%%%%%%%%%%
\newcommand{\ttitle}{Assignment \#2}
% \newcommand{\ttitleEn}{Image Based Biometry Assignment \#2}
\newcommand{\tsubject}{\ttitle}
% \newcommand{\tsubjectEn}{\ttitleEn}
\newcommand{\tauthor}{Tilen Kavčič}
% \newcommand{\tkeywords}{računalnik, računalnik, računalnik}
% \newcommand{\tkeywordsEn}{computer, computer, computer}


%%%%%%%%%%%%%%%%%%%%%%%%%%%%%%%%%%%%%%%%
%	HYPERREF SETUP
%%%%%%%%%%%%%%%%%%%%%%%%%%%%%%%%%%%%%%%%
\hypersetup{pdftitle={\ttitle}}
% \hypersetup{pdfsubject=\ttitleEn}
\hypersetup{pdfauthor={\tauthor}}
% \hypersetup{pdfkeywords=\tkeywordsEn}


 


%%%%%%%%%%%%%%%%%%%%%%%%%%%%%%%%%%%%%%%%
% postavitev strani
%%%%%%%%%%%%%%%%%%%%%%%%%%%%%%%%%%%%%%%%  

\addtolength{\marginparwidth}{-20pt} % robovi za tisk
\addtolength{\oddsidemargin}{40pt}
\addtolength{\evensidemargin}{-40pt}

\renewcommand{\baselinestretch}{1.3} % ustrezen razmik med vrsticami
\setlength{\headheight}{15pt}        % potreben prostor na vrhu
\renewcommand{\chaptermark}[1]%
{\markboth{\MakeUppercase{\thechapter.\ #1}}{}} \renewcommand{\sectionmark}[1]%
{\markright{\MakeUppercase{\thesection.\ #1}}} \renewcommand{\headrulewidth}{0.5pt} \renewcommand{\footrulewidth}{0pt}
\fancyhf{}
\fancyhead[LE,RO]{\sl \thepage} 
%\fancyhead[LO]{\sl \rightmark} \fancyhead[RE]{\sl \leftmark}
\fancyhead[RE]{\sc \tauthor}              % dodal Solina
\fancyhead[LO]{\sc Diplomska naloga}     % dodal Solina


\newcommand{\BibTeX}{{\sc Bib}\TeX}

%%%%%%%%%%%%%%%%%%%%%%%%%%%%%%%%%%%%%%%%
% naslovi
%%%%%%%%%%%%%%%%%%%%%%%%%%%%%%%%%%%%%%%%  


\newcommand{\autfont}{\Large}
\newcommand{\titfont}{\LARGE\bf}
\newcommand{\clearemptydoublepage}{\newpage{\pagestyle{empty}\cleardoublepage}}
\setcounter{tocdepth}{1}	      % globina kazala

%%%%%%%%%%%%%%%%%%%%%%%%%%%%%%%%%%%%%%%%
% konstrukti
%%%%%%%%%%%%%%%%%%%%%%%%%%%%%%%%%%%%%%%%  
\newtheorem{izrek}{Izrek}[chapter]
\newtheorem{trditev}{Trditev}[izrek]
\newenvironment{dokaz}{\emph{Dokaz.}\ }{\hspace{\fill}{$\Box$}}

%%%%%%%%%%%%%%%%%%%%%%%%%%%%%%%%%%%%%%%%%%%%%%%%%%%%%%%%%%%%%%%%%%%%%%%%%%%%%%%
%% PDF-A
%%%%%%%%%%%%%%%%%%%%%%%%%%%%%%%%%%%%%%%%%%%%%%%%%%%%%%%%%%%%%%%%%%%%%%%%%%%%%%%


%%%%%%%%%%%%%%%%%%%%%%%%%%%%%%%%%%%%%%%% 
% define medatata
%%%%%%%%%%%%%%%%%%%%%%%%%%%%%%%%%%%%%%%% 
\def\Title{\ttitle}
\def\Author{\tauthor, matjaz.kralj@fri.uni-lj.si}
\def\Subject{\ttitleEn}
\def\Keywords{\tkeywordsEn}

%%%%%%%%%%%%%%%%%%%%%%%%%%%%%%%%%%%%%%%% 
% \convertDate converts D:20080419103507+02'00' to 2008-04-19T10:35:07+02:00
%%%%%%%%%%%%%%%%%%%%%%%%%%%%%%%%%%%%%%%% 
\def\convertDate{%
    \getYear
}

{\catcode`\D=12
 \gdef\getYear D:#1#2#3#4{\edef\xYear{#1#2#3#4}\getMonth}
}
\def\getMonth#1#2{\edef\xMonth{#1#2}\getDay}
\def\getDay#1#2{\edef\xDay{#1#2}\getHour}
\def\getHour#1#2{\edef\xHour{#1#2}\getMin}
\def\getMin#1#2{\edef\xMin{#1#2}\getSec}
\def\getSec#1#2{\edef\xSec{#1#2}\getTZh}
\def\getTZh +#1#2{\edef\xTZh{#1#2}\getTZm}
\def\getTZm '#1#2'{%
    \edef\xTZm{#1#2}%
    \edef\convDate{\xYear-\xMonth-\xDay T\xHour:\xMin:\xSec+\xTZh:\xTZm}%
}

\expandafter\convertDate\pdfcreationdate 

%%%%%%%%%%%%%%%%%%%%%%%%%%%%%%%%%%%%%%%%
% get pdftex version string
%%%%%%%%%%%%%%%%%%%%%%%%%%%%%%%%%%%%%%%% 
\newcount\countA
\countA=\pdftexversion
\advance \countA by -100
\def\pdftexVersionStr{pdfTeX-1.\the\countA.\pdftexrevision}


%%%%%%%%%%%%%%%%%%%%%%%%%%%%%%%%%%%%%%%%
% XMP data
%%%%%%%%%%%%%%%%%%%%%%%%%%%%%%%%%%%%%%%%  
\usepackage{xmpincl}
% \includexmp{pdfa-1b}

%%%%%%%%%%%%%%%%%%%%%%%%%%%%%%%%%%%%%%%%
% pdfInfo
%%%%%%%%%%%%%%%%%%%%%%%%%%%%%%%%%%%%%%%%  
\pdfinfo{%
    /Title    (\ttitle)
    /Author   (\tauthor)
    % /Subject  (\ttitleEn)
    % /Keywords (\tkeywordsEn)
    /ModDate  (\pdfcreationdate)
    /Trapped  /False
}


%%%%%%%%%%%%%%%%%%%%%%%%%%%%%%%%%%%%%%%%%%%%%%%%%%%%%%%%%%%%%%%%%%%%%%%%%%%%%%%
%%%%%%%%%%%%%%%%%%%%%%%%%%%%%%%%%%%%%%%%%%%%%%%%%%%%%%%%%%%%%%%%%%%%%%%%%%%%%%%

\begin{document}
\selectlanguage{slovene}
\frontmatter
\setcounter{page}{1} %
\renewcommand{\thepage}{}       % preprecimo težave s številkami strani v kazalu
\newcommand{\sn}[1]{"`#1"'}                    % dodal Solina (slovenski narekovaji)

%%%%%%%%%%%%%%%%%%%%%%%%%%%%%%%%%%%%%%%%
%naslovnica
\thispagestyle{empty}%
\begin{center}
    {\large\sc Univerza v Ljubljani\\%
        %      Fakulteta za elektrotehniko\\% za študijski program Multimedija
        %      Fakulteta za upravo\\% za študijski program Upravna informatika
        Fakulteta za računalništvo in informatiko\\%
        %      Fakulteta za matematiko in fiziko\\% za študijski program Računalništvo in matematika
    }
    \vskip 10em%
        {\autfont \tauthor\par}%
        {\titfont \ttitle \par}%
        {\vskip 3em \textsc{IMAGE BASED BIOMERY\\[5mm]}}         % dodal Solina za ostale študijske programe
    %    VISOKOŠOLSKI STROKOVNI ŠTUDIJSKI PROGRAM\\ PRVE STOPNJE\\ RAČUNALNIŠTVO IN INFORMATIKA}\par}%
    %  UNIVERZITETNI  ŠTUDIJSKI PROGRAM\\ PRVE STOPNJE\\ RAČUNALNIŠTVO IN INFORMATIKA}\par}%
    %    INTERDISCIPLINARNI UNIVERZITETNI\\ ŠTUDIJSKI PROGRAM PRVE STOPNJE\\ MULTIMEDIJA}\par}%
    %    INTERDISCIPLINARNI UNIVERZITETNI\\ ŠTUDIJSKI PROGRAM PRVE STOPNJE\\ UPRAVNA INFORMATIKA}\par}%
    %    INTERDISCIPLINARNI UNIVERZITETNI\\ ŠTUDIJSKI PROGRAM PRVE STOPNJE\\ RAČUNALNIŠTVO IN MATEMATIKA}\par}%
    \vfill\null%
    % izberite pravi habilitacijski naziv mentorja!
    %     {\large \textsc{Mentor}: viš. pred./doc./izr. prof./prof. dr. Peter Klepec\par}%
    %    {\large \textsc{Somentor}:  viš. pred./doc./izr. prof./prof. dr.  Martin Krpan \par}%
    {\vskip 2em \large Ljubljana, 2021 \par}%
\end{center}
% prazna stran
%\clearemptydoublepage      % dodal Solina (izjava o licencah itd. se izpiše na hrbtni strani naslovnice)

%%%%%%%%%%%%%%%%%%%%%%%%%%%%%%%%%%%%%%%%
%copyright stran
% \thispagestyle{empty}
% \vspace*{8cm}

% \noindent
% {\sc Copyright}. 
% Rezultati diplomske naloge so intelektualna lastnina avtorja in matične fakultete Univerze v Ljubljani.
% Za objavo in koriščenje rezultatov diplomske naloge je potrebno pisno privoljenje avtorja, fakultete ter mentorja.

% \begin{center}
% \mbox{}\vfill
% \emph{Besedilo je oblikovano z urejevalnikom besedil \LaTeX.}
% \end{center}
% % prazna stran
% \clearemptydoublepage

%%%%%%%%%%%%%%%%%%%%%%%%%%%%%%%%%%%%%%%%
% stran 3 med uvodnimi listi
% \thispagestyle{empty}
% \
% \vfill

% \bigskip
% \noindent\textbf{Kandidat:} Ime Priimek\\
% \noindent\textbf{Naslov:} Naslov diplome v slovenščini\\
% % vstavite ustrezen naziv študijskega programa!
% \noindent\textbf{Vrsta naloge:} npr. Diplomska naloga na univerzitetnem programu prve stopnje Računalništvo in informatika \\
% % izberite pravi habilitacijski naziv mentorja!
% \noindent\textbf{Mentor:} viš. pred. / doc. / izr. prof. / prof. dr. Ime Priimek\\
% \noindent\textbf{Somentor:} isto kot za mentorja

% \bigskip
% \noindent\textbf{Opis:}\\
% Besedilo teme diplomskega dela študent prepiše iz študijskega informacijskega sistema, kamor ga je vnesel mentor. 
% V nekaj stavkih bo opisal, kaj pričakuje od kandidatovega diplomskega dela. 
% Kaj so cilji, kakšne metode naj uporabi, morda bo zapisal tudi ključno literaturo.

% \bigskip
% \noindent\textbf{Title:} Naslov diplome v angleščini

% \bigskip
% \noindent\textbf{Description:}\\
% opis diplome v angleščini

% \vfill



% \vspace{2cm}

% prazna stran
% \clearemptydoublepage

% zahvala
% \thispagestyle{empty}\mbox{}\vfill\null\it%
% \noindent
% Na tem mestu zapišite, komu se zahvaljujete za izdelavo diplomske naloge. Pazite, da ne boste koga pozabili. Utegnil vam bo zameriti. Temu se da izogniti tako, da celotno zahvalo izpustite.
% \rm\normalfont

% prazna stran
% \clearemptydoublepage

%%%%%%%%%%%%%%%%%%%%%%%%%%%%%%%%%%%%%%%%
% posvetilo, če sama zahvala ne zadošča :-)
% \thispagestyle{empty}\mbox{}{\vskip0.20\textheight}\mbox{}\hfill\begin{minipage}{0.55\textwidth}%
% Svoji dragi Alenčici.
% \normalfont\end{minipage}

% prazna stran
% \clearemptydoublepage


%%%%%%%%%%%%%%%%%%%%%%%%%%%%%%%%%%%%%%%%
% kazalo
% \pagestyle{empty}
% \def\thepage{}% preprecimo tezave s stevilkami strani v kazalu
% \tableofcontents{}


% prazna stran
% \clearemptydoublepage

%%%%%%%%%%%%%%%%%%%%%%%%%%%%%%%%%%%%%%%%
% seznam kratic

% \chapter*{Seznam uporabljenih kratic}  % spremenil Solina, da predolge vrstice ne gredo preko desnega roba

% \begin{comment}
% \begin{tabular}{l|l|l}
%   {\bf kratica} & {\bf angleško} & {\bf slovensko} \\ \hline
%   % after \\: \hline or \cline{col1-col2} \cline{col3-col4} ...
%   {\bf CA} & classification accuracy & klasifikacijska točnost \\
%   {\bf DBMS} & database management system & sistem za upravljanje podatkovnih baz \\
%   {\bf SVM} & support vector machine & metoda podpornih vektorjev \\
%   \dots & \dots & \dots \\
% \end{tabular}
% \end{comment}

% \noindent\begin{tabular}{p{0.1\textwidth}|p{.4\textwidth}|p{.4\textwidth}}    % po potrebi razširi prvo kolono tabele na račun drugih dveh!
%   {\bf kratica} & {\bf angleško}                             & {\bf slovensko} \\ \hline
%   {\bf CA}      & classification accuracy               & klasifikacijska točnost \\
%   {\bf DBMS} & database management system & sistem za upravljanje podatkovnih baz \\
%   {\bf SVM}   & support vector machine              & metoda podpornih vektorjev \\
% %  \dots & \dots & \dots \\
% \end{tabular}


% prazna stran
% \clearemptydoublepage

%%%%%%%%%%%%%%%%%%%%%%%%%%%%%%%%%%%%%%%%
% povzetek
% \addcontentsline{toc}{chapter}{Povzetek}
% \chapter*{Povzetek}

% \noindent\textbf{Naslov:} \ttitle
% \bigskip

% \noindent\textbf{Avtor:} \tauthor
% \bigskip

% %\noindent\textbf{Povzetek:} 
% \noindent V vzorcu je predstavljen postopek priprave diplomskega dela z uporabo okolja \LaTeX. Vaš povzetek mora sicer vsebovati približno 100 besed, ta tukaj je odločno prekratek.
% Dober povzetek vključuje: (1) kratek opis obravnavanega problema, (2) kratek opis vašega pristopa za reševanje tega problema in (3) (najbolj uspešen) rezultat ali prispevek magistrske naloge.

% \bigskip

% \noindent\textbf{Ključne besede:} \tkeywords.
% prazna stran
% \clearemptydoublepage

%%%%%%%%%%%%%%%%%%%%%%%%%%%%%%%%%%%%%%%%
% abstract
% \selectlanguage{english}
% \addcontentsline{toc}{chapter}{Abstract}
% \chapter*{Abstract}

% % \noindent\textbf{Title:} \ttitleEn
% \bigskip

% \noindent\textbf{Author:} \tauthor
% \bigskip

%\noindent\textbf{Abstract:} 
% \noindent This sample document presents an approach to typesetting your BSc thesis using \LaTeX.
% A proper abstract should contain around 100 words which makes this one way too short.
% \bigskip

% \noindent\textbf{Keywords:} \tkeywordsEn.
% \selectlanguage{slovene}
% prazna stran
% \clearemptydoublepages

%%%%%%%%%%%%%%%%%%%%%%%%%%%%%%%%%%%%%%%%
\mainmatter
\setcounter{page}{1}
\pagestyle{fancy}

\chapter{Introduction}
We look at face recognition on WIDER FACE database using convolutional neural networks (CNNs).

\chapter{Process}

\section{Data acquisition}
We selected the WIDER FACE dataset and extracted the images and annotations.
For easier training and validation we picked out a subset of the images. More specifically we selected 996 training images from the demonstration category and 248 validation images from the same category.

\section{Preprocessing}
For speed we tryed to minimize the resolution of the images without hampering training. Image size, for this reason, was set to minimum 224x224.
We use `ImageDataGenerator` to rescale the images.
We seperate the images into training and validation sets. The code is shown in appendix \ref{appendix:Preprocessing}.

\section{The model}
We created a base model from the pre-trained CNN model MobileNet V2. This model was created from the training data commpiled in previous steps.
The model was developed at Google and was pre-trained on the ImageNet dataset, a large dataset of 1.4M images and 1000 classes of web images.

Out base model is a 2D Convolution network (32 nodes, 3 Kernel size, Activation Function). We set the probability of each non-contributing node being dropped to 20\%. We use the Softmax activation function. We set Epochs to 10 (model trained in 10 iterations). The code is shown in appendix \ref{appendix:model}.

\section{Evaluation}
We visualise the learing curves of the model. They are shown below \ref{res}. The accuracy of the model after 10 epochs is 0.802. The code is shown in appendix \ref{appendix:Visualisation}.

\begin{figure}[ht]
    \begin{center}
        \includegraphics[width=0.8\textwidth]{images/download (1).png}
    \end{center}
    \caption{Training and valuation accuracy.}
    \label{res}
\end{figure}




% \cleardoublepage
% \addcontentsline{toc}{chapter}{Literatura}
% \bibliography{literatura}
% \bibliographystyle{plainnat}

\begin{appendices}

\chapter{Programska koda}
\section{Preprocessing}
\label{appendix:Preprocessing}
\begin{lstlisting}
IMAGE_SIZE = 224
BATCH_SIZE = 5

# rescales the images
data_generator = tf.keras.preprocessing.image.
    ImageDataGenerator(
        rescale=1. / 255,
        validation_split=0.2)

train_generator = data_generator.flow_from_directory(
    base_dir,
    target_size=(IMAGE_SIZE, IMAGE_SIZE),
    batch_size=BATCH_SIZE,
    subset='training')

val_generator = data_generator.flow_from_directory(
    base_dir,
    target_size=(IMAGE_SIZE, IMAGE_SIZE),
    batch_size=BATCH_SIZE,
    subset='validation')

for image_batch, label_batch in train_generator:
    break

print(train_generator.class_indices)

labels = '\n'.join(sorted(
    train_generator.class_indices.keys())
)

with open('labels.txt', 'w') as f:
    f.write(labels)
\end{lstlisting}

\section{The model}
\label{appendix:model}
\begin{lstlisting}
IMG_SHAPE = (IMAGE_SIZE, IMAGE_SIZE, 3)
base_model = tf.keras.applications.
                MobileNetV2(input_shape=IMG_SHAPE,
                            include_top=False,
                            weights='imagenet')
base_model.trainable = False
model = tf.keras.Sequential([
    base_model, 
    tf.keras.layers.Conv2D(32, 3, activation='relu'),  
    tf.keras.layers.Dropout(0.2),  
    tf.keras.layers.GlobalAveragePooling2D(),  
    tf.keras.layers.Dense(3, activation='softmax')  
])
model.compile(optimizer=tf.keras.optimizers.Adam(),  
              loss='categorical_crossentropy', 
              metrics=['accuracy'])  
model.summary()
print('Number of trainable variables = {}'.
        format(len(model.trainable_variables)))
epochs = 10
history = model.fit(train_generator,
                    epochs=epochs,
                    validation_data=val_generator)
\end{lstlisting}

\section{Visualisation}
\label{appendix:Visualisation}
\begin{lstlisting}
acc = history.history['accuracy']
val_acc = history.history['val_accuracy']

plt.figure(figsize=(8, 8))
plt.subplot(2, 1, 1)
plt.plot(acc, label='Training Accuracy')
plt.plot(val_acc, label='Validation Accuracy')
plt.legend(loc='lower right')
plt.ylabel('Accuracy')
plt.ylim([min(plt.ylim()), 1])
plt.title('Training and Validation Accuracy')
print(val_acc)
\end{lstlisting}

\end{appendices}

\end{document}
